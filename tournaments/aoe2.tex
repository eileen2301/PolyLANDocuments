\documentclass{article}
\usepackage[utf8]{inputenc}

\makeatletter
\newcommand*{\rom}[1]{\expandafter\@slowromancap\romannumeral #1@}
\makeatother

\title{Age of Empires 2: Forgotten Empires}
\author{ForTheSwarm}
\date{2019-02-23}

\usepackage{natbib}
\usepackage{graphicx}
\usepackage{hyperref}
\usepackage{enumitem}

\begin{document}

\makeatletter
\begin{titlepage}
\centering
\includegraphics[scale=0.075]{../img/GECo.png}\\
\vspace{1.2cm}
\LARGE \@title\\ Tournament Rules\\ \normalsize by \@author\\ \@date
\end{titlepage}
\makeatother


\clearpage

\tableofcontents
\clearpage

\section{General Tournament Rules}
Read the general tournament rules of the current event.\\
All the general rules apply to this tournament.

\section{Game Version}
The Age of Empires \rom{2} tournament uses Age of Empires \rom{2}: Forgotten Empires (short ``AoFE''). This is the last game version that does not depend on additional external software and can be played in-LAN, while having support for newer windows versions. We will \textbf{not} use the HD version or Steam version.

\section{Setup}
Only use the game version provided by us to avoid version mismatch problems. There will be a step-by-step manual to set up the game and resolve common problems if it doesn't work out of the box.

Please go through the manual and test a multiplayer match hosting and non-hosting with other people in the LAN \textbf{before} the tournament starts. The tournament admin might be able to help you with problems.

If you are unsure about whether your hardware and operating system combo allows you to play AoFE, drop by at our office before the lanparty starts, so we can test it and resolve problems. Remember to re-check your firewall settings etc. when you arrive at the lanparty.

\section{Procedure}
\subsection{Hosting}
The team named first in the pairing hosts the game.
\subsection{Result}
The losing team announces the result to the tournament admin, also stating whether they ran into time limit.
\subsection{Map}
The map for each game is announced at the beginning of each Swiss Style round by the tournament admin.

\newpage

\section{Settings}
\subsection{Map pool}
Maps chosen to be not water-divided, no special modus (lack of single resource or town center) and not too difficult for mediocre players. This essentially tries to reduce the number of matches which run into time limit and make the tournament fun for all participants.
\begin{itemize}
\item Arabia
\item Coastal
\item Continental
\item Highland
\item Mongolia
\item Yucatan
\end{itemize}

\subsection{Player settings}
All players must chose a different color. Civilization random, unless both teams agree to pick. Tournament admin can still force random if civ picking takes too long (eg. frequent adjustment to enemy civ combo by both teams).

\subsection{Game Settings}
\begin{itemize}[noitemsep]
\item Game Type: Random map
\item Size: normal
\item Difficulty: Normal
\item Ressources: Standard
\item Population limit: 200
\item Reveal Map: Explored
\item Game speed: Normal
\item Starting Age: Standard
\item Victory: timelimit – 1500 years (2:00h)
\item Team together: yes
\item Lock Teams: yes
\item All Techs: no
\item Lock speed: yes
\item Allow cheats: no
\item Record Game: yes
\end{itemize}

Note that Victory setting of \textit{timelimit} does not allow winning by Wonder or Relic. Also note that using Speed setting \textit{Fast} would reduce the maximum available time limit setting of 2 hours to an effective 50 minutes, which is not acceptable. That's the reason for Speed setting \textit{Normal}.

\newpage

\section{Format}
The tournament will be held in \textbf{Swiss Style} format. Modus will be 2v2, which means a team consists of two participants.

\subsection{Swiss Style}
Swiss Style is detailed in the General Tournament Rules:\\
\url{https://geco.ethz.ch/gtr.pdf}

Generally it's a round-based system which tries to pair similarily ranking teams against each others. There is no early dropout, every team gets to play the same number of rounds. However, not every team necessarily plays against every other team.

The initial pairing is done according to informal player level self-assessments, to ensure matches are interesting to both beginners and experts from the start. Giving a highly incorrect self-assessment (both over- and underestimation) does not increase or decrease the winning chance in Swiss Style tournaments, it merely reduces the fun factor.

The number of rounds will be adjusted to total number of participating teams.


\subsection{Final match}
By design, a Swiss Style tournament does not have a distinctive final match and the two highest-scoring teams didn't necessarily play against each other in the final round.

Depending on number of participating teams and decidedness of final ranking, we may ask the top two teams to play an additional final match against each other, or a showmatch without influence on the tournament scoring. The goal is to have one interesting match with commentator for streaming and on-premises public viewing.

If we can't reach a friendly agreement with all members of the two highest-ranking teams on whether to hold a final match, the outcome of the Swiss Style format is the only and final ranking.

\newpage
\section{General rules}
\begin{itemize}
\item No cheating of any kind.
\item No spamming the chat (or taunts) during the game.
\item Pausing the game only if necessary. State the reason with ingame-chat and inform the tournament administrator immediately. Only the player who paused the game may unpause it and only after both teams are ready again - should the opponent team unpause a game it counts as their immediate loss of the game. However, do not hesistate to pause the game if there is any problem.
\item If there is any problem occuring early (unbearable lag, broken game, wrong settings, etc.), immediately pause the game and re-create a new one after resolving the problem. Do not wait until enemy contact.
\item It is assumed that you tested the game on your PC and have played a working multiplayer testgame (at least 3 minutes) both hosting and non-hosting with a real player in our LAN.
\item Play nice.
\end{itemize}

\end{document}







