\documentclass{article}
\usepackage[utf8]{inputenc}

\title{Hearthstone tournament rules}
\author{Oliver "SemigeileSumpfkuh" Hliddal }
\date{February 2019}
\usepackage{natbib}
\usepackage{graphicx}
\usepackage{hyperref}

\begin{document}


 

\begin{document}
\[\[\maketitle
\includegraphics[scale=0.075]{GECo.png}


\clearpage

\tableofcontents
\clearpage


\section{General Tournament Rules}
Read the general tournament rules of the current event.\\
All the general rules apply to this tournament.

\section{Tournament format}
The tournament will have a Round Robin Group phase followed by a Single Elimination Playoff phase. The size of the groups in the first phase and the amount of players who qualify for the playoff phase depend on the number of participants of the tournament and will be announced after signup has closed.



\section{Match format}
The tournament will be played in the standard constructed game mode. This includes all cards that can be played in Ranked standard. Each player must have 4 different decks from 4 different classes ready at the beginning of each match. It is not allowed to use different decks from the same class in one match.
The matches will be played in a BO5 Conquest format. This means that once a game has been won with a class, that class can’t be used anymore in this match.\\


\section{Match rules}
The battle.net status of each player must be set to “online” during the entire match.}

\subsection{Decktracker}
Any type of application that keeps track of cards in your deck or your opponents deck and what cards have been played is strictly forbidden and counts as use of third-party software. The use of such can result in a default loss or a disqualification of the tournament.

\subsection{Before the match}
Every participant of the tournament should write their battle.net-tag in the “Hearthstone” channel. Players should add each other as friends on battle.net. It is not allowed to remove the opponent from the friends list until the match has been finished. The match starts with a challenge in the standard game mode. Once the challenge has been accepted, players are not allowed to enter their collection. The veto process starts once the challenge has been accepted.

\subsection{Veto process}
In the veto process, both players will tell the other player which four classes they have decks of. Then each will veto one of the opponent’s classes. That class can’t be used in this match.

\subsection{During the match}
If an invalid deck (from a class that was not in the veto process, from a banned class, from a class that a game has already been won with by that player or a different deck from a class that the player has already been played a game with) has been selected, the player that made the error has to concede the game and select a valid deck. If this error is done multiple times that player will lose a game and the opponent can select a class from themselves that counts as having won a game with.\\
The challenge should never be canceled until the match is finished. The only exception is a technical issue in which case the opponent must be informed of the reasoning of the cancellation.\\
\\
The first player to win 3 individual games with three different classes wins the match. A game counts as started if both players have selected a valid deck and both players have connected to the game. If a player can’t connect to the match, they must inform their opponent of the problem. \\
A game counts as won if a player puts the health of their opponent to zero or lower or if the opponent concedes the game. \\
If a player has a technical problem or connection issues during the game and loses the game because of it the opponent can decide whether to restart that game or if the game counts as a loss for the player that had the technical issue.

\subsection{After the game}
Each match ends as soon as one player has won 3 three games. The losing player must report the result of the match in the “Tournament” channel on discord. If the battle.net names are different from the names in the tournament bracket the latter should be used.\\

\end{document}
\]
\]
\end{document}