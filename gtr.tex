\documentclass{article}
\usepackage[utf8]{inputenc}

\title{General }
\author{Jan 'Icy' Urech \& Oliver Hliddal}
\date{\today}

\usepackage{natbib}
\usepackage{graphicx}
\usepackage{hyperref}
 

\begin{document}
\makeatletter
\begin{titlepage}
\centering
\includegraphics[scale=0.1]{img/PolyLAN_Zurich_black.png}\\
\LARGE \@title  Tournament Rules\\ \normalsize by \@author\\ \@date
\end{titlepage}
\makeatother


\clearpage

\tableofcontents
\clearpage


\section{Specific rules}
For all tournaments their specific rules apply. You can find them on their tournament page.



\section{Participation} 
To participate in a tournament you have to..
\begin{itemize}
	\item be registered to the according event.
	\item have paid the entry fee.
	\item be checked in at the event.
	\item be signed up before the sign up period ends.
	\item not already be signed up for another main tournament if the tournament you want to sign up is a also a main tournament.
\end{itemize}

\section{Tournament organizer}
The tournament organizer is one or two designated members of the Admin team. They update the scores and are the contact person in case of a dispute. Please do not hesitate to contact the tournament organizer if you have any questions.

\section{Teams}
Each team has to be full to participate in a tournament. If you want to use a substitute/additional player in your team contact the tournament organizer.

\section{Team captain}
The team captain is responsible for the whole team. They're in charge of
communication with other teams and the tournament organizer.
The team captain is the owner of the team, as indicated on our website.
They can only be changed before the tournament starts. After each match, the team captain of the losing team has to announce the result to the tournament organizer.


\section{Meeting}
At the start of every tournament there will be a meeting with all team captains and the tournament organizer. It is crucial to attend this meeting since a lot of important information for the respective tournament will be communicated at that meeting.


\section{Timetable}
Matches have to be played at the designated time. Time changes are only possible if the tournament schedule allows for a change and the change is in agreement with both the tournament organizer and the opponent.\\
Main tournaments always have priority. In case of delays in one tournament the main tournament has to be played over a side tournament.
You have to be ready to play your matches during the hours indicated on the timetable.\\
This applies to:
\begin{itemize}
	\item the match times for the group stage
	\item the start of the team briefing for the team captain
	\item the start of the playoff phase
	\item the start of the finals
\end{itemize}
Showing up late can lead to a game($>$5min) or even a match loss($>$15min).


\section{Seeding}
Tournament seedings will either be done at random or by the tournament organizer.\\
On some sign ups you will be asked for a estimate of your skill level in order to estimate the strength of your team.
Please provide an honest estimate this will make the tournament better for you and other participants.


\section{Tournament modes}
Most tournaments have a fixed tournament mode, but the optimal group and bracket sizes will be chosen depending on the amount of teams participating at the event.\\
All tournaments will be held in one of the following modes:

\subsection{Single-Elimination}
In Single-Elimination each team is eliminated when it loses a match.\\


\subsection{Double-Elimination}
In Double-Elim each team is eliminated when it lost 2 matches.\\
After losing the first match a team moves to the loser-bracket.\\
In some tournaments in the finals the team coming from the upper-bracket is given some kind of advantage.\\
e.g. head start of one match in a best of 3

\subsection{Round Robin}
Every team in a group plays against every other team from the same group.
The winner is defined according to a predefined set of criteria.
If no winner can be found with the first criteria the next one is considered and so on.\\
(e.g. Wins, RoundsWon, RoundsLost)

\subsection{Swiss}
The Swiss-system is similar to Round Robin. But matchups are always formed according to the previous losses and wins. Matching teams with similar success. After a certain number of rounds the winner is defined as in Round Robin. The number of rounds is determined by the number of teams participating.


\subsection{Groups/Playoffs}
First all teams are playing in Groups(see %Swiss/
Round Robin).\\
Then the top x teams of a Group move on to the Playoffs. The exact number gets determined after signup ends.\\
The Playoffs are held as Single-Elimination.


\section{Player Behavior}
\subsection{Cheating}
Any kind of cheating is strictly forbidden.\\
Abusing bugs on purpose, using any third-party software that gives you an unfair advantage and ghosting in any way are also considered cheating.

\subsection{Flaming}
Don't be toxic!\\
The event should be fun for everyone.


\subsection{Tournament organizer instructions}
Always follow the instructions of the tournament organizers.



\end{document}